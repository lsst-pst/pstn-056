\documentclass[PST,authoryear,toc]{lsstdoc}
% GENERATED FILE -- edit this in the Makefile
\newcommand{\lsstDocType}{PSTN}
\newcommand{\lsstDocNum}{056}
\newcommand{\vcsRevision}{08c38ae-dirty}
\newcommand{\vcsDate}{2024-09-07}


% Package imports go here.

% Local commands go here.

%If you want glossaries
%\input{aglossary.tex}
%\makeglossaries

\title{Survey Cadence Optimization Committee’s Phase 3 Recommendations}

% This can write metadata into the PDF.
% Update keywords and author information as necessary.
\hypersetup{
    pdftitle={Survey Cadence Optimization Committee’s Phase 3 Recommendations},
    pdfauthor={federica bianco (she/her/hers)},
    pdfkeywords={}
}

% Optional subtitle
% \setDocSubtitle{A subtitle}

\author{%
federica bianco (she/her/hers)
}

\setDocRef{PSTN-056}
\setDocUpstreamLocation{\url{https://github.com/lsst-pst/pstn-056}}

\date{\vcsDate}

% Optional: name of the document's curator
% \setDocCurator{The Curator of this Document}

\setDocAbstract{%
The final recommendation of the SCOC for the Survey Strategy prior to the start of LSST
}

% Change history defined here.
% Order: oldest first.
% Fields: VERSION, DATE, DESCRIPTION, OWNER NAME.
% See LPM-51 for version number policy.
\setDocChangeRecord{%
  \addtohist{1}{YYYY-MM-DD}{Unreleased.}{federica bianco (she/her/hers)}
}


\begin{document}

% Create the title page.
\maketitle
% Frequently for a technote we do not want a title page  uncomment this to remove the title page and changelog.
% use \mkshorttitle to remove the extra pages

% ADD CONTENT HERE
% You can also use the \input command to include several content files.

\appendix
% Include all the relevant bib files.
% https://lsst-texmf.lsst.io/lsstdoc.html#bibliographies
\section{References} \label{sec:bib}
\renewcommand{\refname}{} % Suppress default Bibliography section
\bibliography{local,lsst,lsst-dm,refs_ads,refs,books}

% Make sure lsst-texmf/bin/generateAcronyms.py is in your path
\section{Acronyms} \label{sec:acronyms}
\addtocounter{table}{-1}
\begin{longtable}{p{0.145\textwidth}p{0.8\textwidth}}\hline
\textbf{Acronym} & \textbf{Description}  \\\hline

AGN & Active Galactic Nuclei \\\hline
B & Byte (8 bit) \\\hline
DDF & Deep Drilling Field \\\hline
DESC & Dark Energy Science Collaboration \\\hline
DM & Data Management \\\hline
DR11 & Data Release 11 \\\hline
DR2 & Data Release 2 \\\hline
GW & Gravitational Wave \\\hline
JPL & Jet Propulsion Laboratory (DE ephemerides) \\\hline
LIGO & Laser Interferometer Gravitational-Wave Observatory \\\hline
LMC & Large Magellanic Cloud \\\hline
LPM & LSST Project Management (Document Handle) \\\hline
LSST & Legacy Survey of Space and Time (formerly Large Synoptic Survey Telescope) \\\hline
M1 & primary mirror \\\hline
M2 & Secondary Mirror \\\hline
M3 & tertiary mirror \\\hline
MAF & Metric Analysis Framework \\\hline
MC & Monte-Carlo (simulation/process) \\\hline
MMA & Multi Messenger Astronomy \\\hline
MW & Milky Way \\\hline
NES & North Ecliptic Spur \\\hline
PCW & Project Community Workshop \\\hline
PST & Project Science Team \\\hline
PSTN & Project Science Technical Note \\\hline
RA & Risk Assessment \\\hline
SC & Science Collaboration \\\hline
SCOC & Survey Cadence Optimization Committee \\\hline
SCP & South Celestial Pool \\\hline
SMC & Small Magellanic Cloud \\\hline
SN & SuperNovae \\\hline
SRD & LSST Science Requirements; LPM-17 \\\hline
TVS & Transients and Variable Stars Science Collaboration \\\hline
ToO & Target of Opportunity \\\hline
WFD & Wide Fast Deep \\\hline
YSO & Young Stellar Object \\\hline
photo-z & photometric redshift \\\hline
\end{longtable}

% If you want glossary uncomment below -- comment out the two lines above
%\printglossaries





\end{document}
