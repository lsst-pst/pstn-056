\documentclass[PST,authoryear,toc,lsstdraft]{lsstdoc}
% GENERATED FILE -- edit this in the Makefile
\newcommand{\lsstDocType}{PSTN}
\newcommand{\lsstDocNum}{056}
\newcommand{\vcsRevision}{864d65e-dirty}
\newcommand{\vcsDate}{2024-05-24}


\defcitealias{PSTN-055}{PSTN-055}
\defcitealias{PSTN-053}{PSTN-053}
\defcitealias{LPM-17}{SRD}

\newcommand{\tbd}[1]{{\color{red} #1}}
\newcommand{\new}[1]{{\color{blue} #1}}
\newcommand{\incomplete}[1]{{\color{blue} [PROVISIONAL RECOMMENDATION ] #1}}
\newcommand{\pz}{{Photo-$z$}}
%\newcommand{\ie}{{\it i.e.}}
%\newcommand{\eg}{{\it e.g.}}
\newcommand{\opsim}{\texttt{OpSim}}
\newcommand{\baseline}[1]{{\texttt{baseline\_v#1}}}
% Package imports go here.
\usepackage{placeins}
%\usepackage[printwatermark]{xwatermark}
%\usepackage{xcolor}
%\usepackage{graphicx}
%\newwatermark[allpages,color=red!50,angle=45,scale=3,xpos=0,ypos=0]{DRAFT}
\usepackage{overpic}

% Local commands go here.

%If you want glossaries
%\input{aglossary.tex}
%\makeglossaries

\title{Survey Cadence Optimization Committee’s Phase 3 Recommendations}

% This can write metadata into the PDF.
% Update keywords and author information as necessary.
\hypersetup
{pdftitle={Survey Cadence Optimization Committee’s Phase 3 Recommendations},
    pdfauthor={Federica B. Bianco and the SCOC},
    pdfkeywords={}
}

% Optional subtitle
% \setDocSubtitle{A subtitle}

\author{%
the SCOC
}

\setDocRef{PSTN-056}
\setDocUpstreamLocation{\url{https://github.com/lsst-pst/pstn-056}}

\date{\vcsDate}

% Optional: name of the document's curator
% \setDocCurator{The Curator of this Document}

\setDocAbstract{%
The final recommendation of the SCOC for the Survey Strategy prior to the start of LSST
}

% Change history defined here.
% Order: oldest first.
% Fields: VERSION, DATE, DESCRIPTION, OWNER NAME.
% See LPM-51 for version number policy.
\setDocChangeRecord{%
  \addtohist{1}{YYYY-MM-DD}{Unreleased.}{Federica Bianco and the Survey Cadence Optimization Committee}
}


\begin{document}

% Create the title page.
\maketitle
% Frequently for a technote we do not want a title page uncomment this to remove the title page and changelog.
% use \mkshorttitle to remove the extra pages

% ADD CONTENT HERE
% You can also use the \input command to include several content files.

%%%% THE SECTIONS %%%%%%
\section{Introduction}

With an unprecedented engagement of the scientific community at large, the Vera C. Rubin Observatory (hereafter Rubin) has designed a process of incremental improvements to the survey strategy to maximize the overall scientific throughput of the Legacy Survey of Space and Time (LSST). The high-level requirements for the LSST are set by four science pillars: probing dark energy and dark matter, building an unprecedented inventory of the Solar System, mapping the Milky Way and Local Volume, and exploring the transient universe. 

These requirements are described in \cite{LPM-17} ---hereafter the Science Requirements Document, or \citetalias{LPM-17}---, but significant flexibility remains in survey cadence within these requirements. The optimization of the survey strategy process is aimed at maximizing science for the four science pillars and increasing the portfolio of LSST science by tuning the survey strategy and cadence within the SRD requirements (\citetalias{LPM-17}).

As part of this process, the Survey Cadence Optimization Committee (SCOC) was set up by Rubin's Science Advisory Committee in 2018 to solicit, review, and integrate community feedback and make recommendations for the implementation of the LSST survey strategy to the Director of Operations. This document constitutes the third SCOC recommendation, resulting from the Phase 3 process of survey design which started in January 2023, after the delivery of the Phase 2 recommendation  (\citealt{PSTN-055} ---hereafter \citetalias{PSTN-055}--- and the baseline simulation \texttt{baseline\_v3.0}. \cite{PSTN-056} (this document) is planned to be the last recommendation for the LSST as a whole before the start of LSST. However, the SCOC will refine the plan for Y1 in particular and the LSST in general in light of commissioning outcomes; reviews of the survey strategy will continue throughout the 10-year survey,  renewing its recommendation on an annual basis (\autoref{sec:next}). 

The Phase 3 recommendation (this document) responds directly to the questions left open in Phase 2 (\citetalias{PSTN-055}) and updates and refines previous recommendations (\citetalias{PSTN-055} and  \citetalias{PSTN-053}) The present document generally does not reiterate previous recommendations that have not changed.




The document is structured as follows.
%The open questions identified in \citetalias{PSTN-055} Section 4  are included here for the reader's convenience in \autoref{sec:openquestions}. The SCOC recommendations relative to those questions follow each question. 
Each of the points left for deliberation in  \citetalias{PSTN-055} is discussed in \autoref{sec:openquestions}. 
Additional changes to the survey strategy are described in \autoref{sec:additional} and changes to the simulations, beyond the content of this recommendation, in \autoref{sec:opsimchanges}. 
The current recommendation is summarized in \autoref{sec:summary} and the \baseline{4.0} simulations are described in \autoref{sec:baseline4_0}.
The planned activities of the SCOC in Operations, and topics that the SCOC should focus on in the next round of deliberations, including the process of interaction with the community and iterative optimization of the LSST during Operations follow in \autoref{sec:next}.
This document includes definitions of acronyms and terms used in \autoref{sec:acronyms}.

The simulations discussed in this recommendation are available at \url{https://usdf-maf.slac.stanford.edu}

In the spirit of reproducibility, a notebook that generates the figures contained in this notebook is available at \url{https://github.com/lsst-pst/pstn-056/blob/main/notebooks/pstn-056-figures.ipynb}.

\clearpage
%In \autoref{sec:process} we describe .

\section{Executive Summary of the Phase 3 recommendation}


To help the reader parse the content that follows, we note that the LSST survey is actually an ensemble of surveys. It includes a main survey, known as Wide Fast Deep (WFD), which by SRD ``goal'' (minimum) requirement should receive more than 825 (750) observations and cover at least 18,000 (16,000) square degrees, and a Galactic Plane (and Bulge) survey. Furthermore, a WFD low-dust region is defined with limits $-70^o \leq \mathrm{Dec} \leq +12.5^o$ for 
$\mathrm{RA} \sim   7-18 h$ and $-72^o \leq \mathrm{Dec}\leq +3^o$
for $0 \lesssim \mathrm{RA} \lesssim 7$ h and $18 h \lesssim \mathrm{RA} \lesssim 24 h$, with the addition of the Virgo cluster, as of \citetalias{PSTN-055}.  Special regions in the LSST include the LMC, SMC, South Celestial Pole (SCP), and North Ecliptic Spur (NES). Microsurveys, or surveys performed with a different cadence from the WFD and that require $<3\%$ of the LSST time can be performed and \citetalias{PSTN-055} committed to a twilight Near-Earth-Objects (NEO) microsurvey in the first year of LSST, Y1, and to accept further proposals for future microsurveys after the start of LSST. A Target of Opportunity program was recommended for up to $3\%$ of LSST time in \citetalias{PSTN-055}.

Notable updates from previous recommendations, and corresponding changes to the baseline, include updating system throughputs,  expectations for engineering time (particularly in Y1), and slew performance; tweaking the filter balance in response to throughput changes; refining observing choices in the Galactic Plane, Bulge, and LMC/SMC/SCP; defining the implementation plan for the ToO program; recommending single visit exposures over visits implemented in ``snaps''; investigating new rolling strategy options; refining the DDF observing plans.


\subsection{Note on how to read the SCOC plots}

The SCOC typically reviews the outcome of metrics measuring system performance and science outcomes built within the Metric Analysis Framework \citep{2014SPIE.9149E..0BJ}, hereafter MAFs. MAFs are reviewed across multiple simulations of the 10-year sequence of LSST observations, each simulation referred to as an \opsim\, to compare scientific performance. In this document, you will see sky maps measuring quantities by healpixels (\eg, number of visits, or any MAF). The typical sky pixelization that underlies the metric calculations the SCOC reviews is 128 sides healpixels (covering an area $\sim 0.2\mathrm{deg}^2$), although for particularly computationally intense MAFs this can be turned down to 64 or 32. You will see metric plots in the following styles: 



{\bf Heatmaps}: \autoref{fig:heatmap} -
the divergent color scheme shows improvements in metrics in blue and drops in performance in red. Note that different heat maps may show different ranges in the color scheme, but the SCOC typically considers changes of more than a few percent to be significant and less than a few percent to be in the noise. One of the \opsim s is chosen as a reference and the corresponding column will look neutral in color.

\begin{figure}
  \centering
    \includegraphics[width=0.95\textwidth]{figures/v1-v34heatmap.png}
\caption{A heatmap produced by the Observing Strategy team for the SCOC. The plot compares the performance of selected metrics across baseline simulations from \baseline{1.X} (original survey strategy) through \baseline{3.4}. In these plots, an \opsim\ is selected as the reference and the performances of all other \opsim s are shown relative to that. That is: the reference \opsim\ (\baseline{2.0} in this case) has MAF=1 for all the metrics. Blue colors indicate a positive metric value, \ie\ an improvement. Red colors indicate a performance drop with respect to the reference \opsim. Significant changes in the observing strategy can be seen to impact nearly all metrics (the whole column). For example: the changes between \baseline{3.2} and \baseline{3.3} are due to the updated filter throughput models corresponding to updated plans for mirror coating (discussed in \autoref{sec:filterbalance}). These changes lead to an overall increase in survey depth, resulting in all tracked metrics having the same or better performance (bluer or the same color as the previous column).}
\label{fig:heatmap}
\end{figure}


\FloatBarrier

{\bf Radar plots}: \autoref{fig:radar} - when comparing small numbers of metrics and few \opsim\ we often use radar plots. Each corner of the radar plot corresponds to a metric, and the colored lines inside the plots that join each corner show metric performance. In these plots, the reference \opsim s looks like a N-gone (or N-sided circle). Where the MAF performance shows improvements compared to the reference \opsim\ the point lies outside of the N-gone, where there is a loss, it sits inside. The range of performance changes, so readers should carefully inspect the plot to see the performance scaling going in and out of the N-gone (typical values are 0.9-1.1). 

\begin{figure}
  \centering
    \includegraphics[width=0.95\textwidth]{figures/radarplot.png}

\caption{A radar plot comparing the performance of MAFs under different filter swapping schemes (see \autoref{sec:filterswap}). This plot compares \baseline{3.0} with \baseline{3.2}. All metrics shown perform as well or better in \baseline{3.2} as shown by the orange points laying outside of the blue polygon, except for \texttt{N~YSO}. However, the loss in \texttt{N~YSO} is minimal and not statistically significant. The range of the axis is 0.8 to 1.3, indicating that a point lying in the center would measure a performance 20\% worse than the reference \opsim, and a point on the outer perimeter would indicate a 30\% improvement.}
\label{fig:radar}
\end{figure}


\FloatBarrier

%{\bf Line plots}: \autoref{fig:lineplot} -
%line plots can show multiple \opsim s and metrics at once. The comparison metric will sit as a horizontal line at 1.0 throughout the plot. The regions outside of the 3\% range we typically consider not-significant are shaded in red (for MAF loss) and green (for MAF gains).


%\begin{figure}
%    \centering
%    \includegraphics[width=0.8\linewidth]{figures/lineplot.png}
%    \caption{Line plot showing the performance of a collection of \opsim\ designed for the Milky Way Task Force (see \autoref{sec:galaxy}) and the corresponding performance for a collection of related Microlensing MAFs. The comparison \opsim\ is \baseline{3.3} (where all the MAFs meet at $y=1$.)}
%\label{fig:lineplot}
%\end{figure}
%\FloatBarrier


\clearpage

\section{Resolution of recommendations left incomplete in PSTN-055} \label{sec:openquestions}

\subsection{Swapping filters on the filter wheel}\label{sec:filterswap}
The filter system at Rubin allows five of the six filters ($u$, $g$, $r$, $i$, $z$, and $y$) to be mounted at the same time on the carousel. 
%Filters at the edge of the system transmission will be swapped in and out of the filter wheel throughout the lunar cycle. 
Filters will be swapped in and out of the filter wheel based on sky brightness due to the lunar phase. 
In \citetalias{PSTN-055} \S4, the SCOC recommended further investigation of which filters to swap: 
\begin{quote}
    {[\citetalias{PSTN-055} \S4] ``The SCOC recommends that the investigation of the filter swapping schemes on the filter wheel continue. After the November 2022 SCOC workshop, a few experiments in swapping $u$, $z$, and $y$ instead of $u$ and $z$ were implemented in \texttt{v2.99} simulations. More work is needed to understand the impacts of this decision on the DDFs as well as on the WFD. Filter pairing prescriptions for the observation pairs should also be explored in some more depth.''}
\end{quote}

 Simulations prior to \texttt{v3.0} swapped the $z$ with $u$ based on lunation, as scattered moonlight is blue and impacts observations in $u$-band most significantly. Simulations tagged \texttt{v3.2}  experimented with  swapping $u$ with $z$ or $y$, including putting all of $u$, $z$, and $y$ on rotation. Swapping a filter has two effects: it adds a gap for the period while it is unavailable, and it increases the cadence in that bandpass during the time it is mounted to achieve the final desired number of observations. Increasing the availability of $z$ on the filter wheel produced significant improvements in supernova (SN) cosmology, especially in the Deep Drilling Fields (DDFs), while swapping two filters instead of three improves coverage at short time scales in filters through $z$ with significant benefits for the study of rapid-evolving transients (\eg\ Kilonovae, KN, see \autoref{fig:swapping}). Keeping the $g$, $r$, $i$, and $z$ filters in the camera at all times also reduces the risk of damaging these critical filters during filter swaps.
 
{\bf The SCOC recommends swapping $u$- and $y$-band according to the moon phase. Having the $z$ filter always available produces benefits for SN cosmology while preserving coverage on short timescales. This recommendation is implemented starting in \baseline{3.2}}.

\begin{figure}
  \centering
%\includegraphics[width=0.5\textwidth]{figures/filter_swap_linear.png}
\includegraphics[width=0.9\textwidth]{figures/filter_swap_KNe.png}
%\includegraphics[width=0.45\textwidth]{figures/filter_swap_SNIa.png}
\caption{The impact of swapping different filters on the Rubin system filter wheel according to lunation on time-domain metrics. The reference \opsim\ swaps the $u$ with the $y$ filter, as per the current SCOC recommendation (blue). The performance for \opsim s swapping $u$ with $z$ is shown in orange, and $u$ with $z$ {\it and} $y$ in alternation is shown in green. Gaps in $z$-band are particularly problematic for high-redshift SNIa detections, an effect that is magnified in the Deep Drilling Fields. %Increases in $z$ band cadence can be beneficial for KNe detection, if repeated observations in the same filter within a rapid timeframe is the criteria. 
Swapping three filters ($u$ with alternating $z$ and $y$) increases the length of time gaps between sampling in the same filter, decreasing performance for hour- to day-time scales transients.
Swapping $u$ with $y$ while leaving $z$ mounted on the filter wheel has an overall positive science impact, balancing the needs of rapid transient science with SNIa science in the Deep Drilling Fields. }
\label{fig:swapping}
\end{figure}

\FloatBarrier
\subsection{Filter Balance}\label{sec:filterbalance}
 In \citetalias{PSTN-055}, the SCOC confirmed the recommendation on the filter balance as implemented starting in \baseline{2.0} but left the possibility that:
  
 \begin{quote}
    {[\citetalias{PSTN-055} \S4] ``While the SCOC recommends the filter balance as implemented starting in \baseline{2.0} should not be changed, it is possible that rebalancing the exposure time to compensate for performance and throughput in some filters as compared to others or shortening exposures in filters where the throughput exceeds expectations enabling the collection of more images in that filter (or overall) would lead to enhanced LSST science. The SCOC cannot finalize this recommendation at this time due to missing information about the characteristics of the system-as-built.''}
\end{quote}

Simulations of the survey strategy up to and including \baseline{3.2} use throughput curves assuming mirror coating as Al-Ag-Al respectively for M1-M2-M3. The plan to coat the mirrors was updated in 2023 to Ag-Ag-Ag (or 3xAg), which leads to a $\sim$15-20\% increase in survey efficiency compared to Al-Ag-Al by increasing throughput in all bands redder than $u$, and bringing throughput closer to the design goals as stated in \citetalias{LPM-17}.\footnote{\url{https://community.lsst.org/t/rubin-sim-v1-3-released/7937} and \url{https://github.com/lsst-pst/syseng_throughputs/blob/main/notebooks/SilverVsAluminum.ipynb}.}  \autoref{tab:dm5Agx3} shows the magnitude limit changes associated with the two different coatings for both detector types in the camera.  Ag-Ag-Ag coating increases sensitivity in all LSST bands from $g$ to $y$, but it decreases the throughput in $u$.  As of \baseline{3.3}, all \opsim\ simulations include the Ag-Ag-Ag expected throughput.

The SCOC reviewed the largely positive impact of the new throughput on science cases: nearly all MAFs responded positively to the increase in survey depth (see \autoref{fig:heatmap} and note the significant improvements between \baseline{3.2} and \baseline{3.3} when the new filter transmission curves were introduced). Some system metrics corresponding to \citetalias{LPM-17} requirements show improvements as large as 10\% (\eg\ Parallax uncertainty, see \autoref{fig:parallax}) and some time domain metrics improve by $\sim$20\% (Kilonovae and SN Ia metrics, see \autoref{fig:heatmap}). 
\clearpage

\begin{longtable}{lccccc}
 \\\hline
& Al-Ag-Al E2V &	Al-Ag-Al ITL & &		Ag-Ag-Ag E2V &		Ag-Ag-Ag ITL\\
\hline
$u$	 &	0.0	 &	$-$0.06 &	$\qquad$ &	$-$0.21 &   $-$0.27\\
$g$	 &	0.0	 &	$-$0.04 &	$\qquad$ &	$+$0.06 &	$+$0.02\\
$r$	 &	0.0	 &	$-$0.05 &	$\qquad$ &   $+$0.10 &	$+$0.05\\
$i$	 &	0.0	 &	$-$0.02 &	$\qquad$ &	$+$0.13 &	$+$0.12\\
$z$	 &	0.0	 &	$+$0.01  &	$\qquad$ &	$+$0.15  &	$+$0.15\\
$y$	 &	0.0	 &	$+$0.03  &	$\qquad$ &	$+$0.07  &	$+$0.10\\
\caption{\small{Magnitude limit changes for camera chips acquired from different vendors (E2V and ITL) and different mirror coating choices (Al-Ag-Al and Ag-Ag-Ag). The reference is E2V chips coated with Al-Ag-Al (first column). Positive values indicate deeper limiting magnitudes.
}}\label{tab:dm5Agx3}
\end{longtable}

\begin{figure}[!h]
    \centering    \includegraphics[width=0.75\textwidth]{figures/parallax.png}
\caption{Gains in the metric tracking LSST's median parallax uncertainty (milliarcseconds) at magnitude $r=24$, an LSST \citetalias{LPM-17} system requirement, for different $\texttt{baseline}$ \opsim s, from \baseline{1.x} through \baseline{3.3}, the first simulation with updated system throughput  reflectivity from the Ag-Ag-Ag mirror coating. The improvements in parallax uncertainty between the \baseline{3.2} and \baseline{3.3} \opsim\ come from the increased depth in all bands bluer than $u$. Similar improvements are seen in proper motion uncertainty, also a quantity under \citetalias{LPM-17} requirements. Uncertainties reflect the impact of weather.}
\label{fig:parallax}
\end{figure}





However, the SCOC understands that the throughput loss in $u$-band ($\sim$30\%  loss in coadded depth) would negatively impact science cases, including Photo-z, studies of the Milky Way halo, and Lyman Break Galaxies (LBGs, identified as $u$-band dropouts at redshift $z\sim3$). While the overwhelming majority of the MAF metrics available to the SCOC responded positively to the updated throughput, we are aware, as always, that these may not provide an exhaustive picture of the science outcomes. Therefore, guided by experts in the community, we explored ways to reduce the $u$ band magnitude decrease while preserving the benefit of increased throughput in redder bands. 
We tracked the performance of Photo-z, as characterized in \citealt{Graham_2017}, by assessing the variance and bias in \pz\ at redshifts $z\lesssim{3}$ (\autoref{fig:pz}). \pz\ is sensitive to $u$ band depth at redshift $z\geq 2$ due to decreased power to identify Lyman break galaxies photometrically. We expect that recovering \pz\ performance is a good indicator of recovering performance for other science cases sensitive to $u$ band depth for which we do not have detailed metrics. \pz\ performance, along with a large set of MAFs, was run against a set of \opsim s  that progressively changed the exposure time and the number of exposures in $u$-band (see \autoref{fig:uband})\footnote{\url{https://community.lsst.org/t/release-of-v3-4-simulations/8548}}. 

{\bf The SCOC recommends:} 

\begin{itemize} 
\item {\bf an increase of the exposure time in $u$-band to 38 seconds per visit}
\item {\bf an increase of the number of $u$-band visits of 10\% compared to \baseline{3.0}}
\item {\bf an identical decrease of 0.8 second in exposure time per visit in all other bands to compensate for the added time in $u$-band.} 
\end{itemize}


This roughly restores the $u$ band depth of LSST in \baseline{3.0} with minimal impact on LSST science metrics. As a science case that is representative of those sensitive to $u$-band depth, these changes recover performance on \pz\ at redshift $z\sim2$, where the impact of the $u$-band throughput loss was most significant, while maintaining the performance improvement on \pz\ at low redshift afforded by the increased depth of LSST in all other bands. Furthermore, these changes minimally impact other science cases tracked by MAFs. 

Because more science cases generally respond better to increasing the number of images, over increasing the exposure time to achieve the same depth, the added $u$-band time should be obtained by decreasing (minimally) the exposure time in other bands rather than decreasing the number of images.

\textbf{ We note that this recommendation is subject to ongoing feasibility studies by the Rubin Data 
Management team.}

\begin{figure}
  \centering
 \includegraphics[height=0.6\textwidth]{figures/photo-z.png}

\caption{The effect of changes in $u$-band 10-year depth on \pz\ Robust Standard deviation as a function of redshift $z$, as measured in \cite{Graham_2017}. The dashed line represents the \citetalias{LPM-17} requirements on \pz. The black solid curve is the \baseline{3.2}, the latest baseline before the filter transmission curves were updated in the Rubin simulation system, the colored curves represent the \pz\ Robust Standard Deviation varying the  $u$-band exposure time between $30\leq u_{expt} \leq45$~seconds and the number of exposures in $u$ band between $1.0\times ns \leq N_u \leq 1.1\times ns$, where $ns$ is the number of $u$-band exposures in \baseline{3.2}. Note that, with the caveat that sampling uncertainties are large at $z>1.5$, with the Ag-Ag-Ag transmission curves we note an improvement in \pz\ at low $z$ associated with increased depth in bands redder than $u$, but a degradation at $z>1.5$. The current filter balance recommendation (closely reflected by the red line in this plot) more than recovers performance at high $z$ while preserving the low $z$ gains. A similar impact is seen in \pz\ bias.}
\label{fig:pz}
\end{figure}

\begin{figure}
\centering
   \includegraphics[width=0.95\textwidth]{figures/u_band_scoc_heatmap.png}
\caption{ A standard set of science and system MAFs metrics as a function of changing the exposure time ($27\leq u_{expt}\leq 45$ seconds) and the fraction of exposures in the $u-$band ($0.9\times ns\leq N_u \leq1.2\times ns$). The metrics are normalized with respect to a simulation with $u_{expt}$= 30 seconds and $N_u = 1.0\times ns$. Three additional columns on the left show: \texttt{v\_3.2} (\baseline{3.2}, pre-filter-throughput update, notably generally worse) and \texttt{v3.3} (\baseline{3.3}, which follows exactly the same observing strategy as \baseline{3.2} but includes throughput updates) and $u~38s~1*$ where the exposure time of all other bands is adjusted to compensate for extra time spent in $u$ (whereas in all other simulations shown in this plot the exposure time is kept at 30 seconds). The SCOC recommends an adjustment of the exposure in all bands ($\sim$29 instead of 30 seconds) and this is implemented in all simulations starting with \texttt{v3.5}.}
\label{fig:uband}
\end{figure}
\FloatBarrier

\subsection{Rolling}\label{sec:rolling}
In a rolling strategy, instead of distributing visits uniformly on the WFD footprint, the sky is split into regions that alternate between high- and low-intensity monitoring. In \citetalias{PSTN-055}, the SCOC recommended the implementation of a rolling strategy for the LSST WFD at a strength of 0.9\footnote{This number represents the fraction of the visits that the scheduler attempts to place in the high-activity rolling region. However, the resulting visit distribution is more uniform (75--80\% in high-activity regions, 25--20\% in low ones) due to competing requirements (\eg , filter balance, minimum number of observations per pointing per year in each filter to produce templates, weather, etc...)} 
with the sky split into two rolling regions constituted by four longitudinal stripes. The primary drivers for this recommendation are time-domain science, including the exploration of the transient and variable sky and SN cosmology. Rolling as described decreases the median time gaps compared to a no-rolling implementation of LSST:
distributing the $\sim$800 visits per pointing evenly into 10 seasons results in a median revisit time per pointing of about 4.5 nights, while rolling can increase the cadence on the areas of sky closer to 2.5 nights.\footnote{The reader is reminded that each pointing receives two or three visits per night. The time gaps reported here are for inter-night observations.}

 However, concerns were raised by the DESC, and seconded by other groups such as the Galaxies SC, regarding the lack of uniformity in the distribution of depths across the survey in planned yearly data releases for intermediate years between 1 and 10 compared to a no-rolling strategy \citep[see discussion in][which noted this as a potential future concern before the adoption of a rolling cadence as the baseline]{2022ApJS..259...58L}.  These concerns highlighted the negative impact that rolling induces on the cosmological analysis conducted with static-sky probes due to a decreased uniformity of the data releases, which has been shown to cause several significant issues for cosmological large-scale structure analyses \citep[\eg ,][]{2022PhRvD.105b3520A,2023JCAP...07..044B}.  
 This uniformity challenge could be addressed after data collection by selecting and limiting the number of images going into a data release per field (including these data in future data releases) to achieve higher uniformity or by a ``renoising'' step. These data management solutions are, at the moment, unscoped and do not fall under the current requirements of the Rubin DM deliverables.

The official Phase 2 recommendation stated that:
\vspace{-0.5cm}
\begin{quote}
     {[\citetalias{PSTN-055} \S4] 
     ``The current SCOC recommendation is to implement a rolling cadence with a half-sky rolling scheme and a 0.9 rolling weight. However, rolling impacts the uniformity of static data releases which, as experts in the community have highlighted, is necessary for static sky science in general and cosmology in particular. This issue may be resolved or mitigated at the software level in the creation of coadds and catalogs, rather than at the scheduler level. The community should specify the desired and necessary requirements for uniformity to enable the exploration of data processing solutions to this problem. Depending on the feasibility of a solution to ensure sufficient uniformity, the SCOC recommendation on rolling may be re-evaluated.'' }

 \end{quote}

 With the goal of quantifying the necessary uniformity to enable cosmological results at certain key data releases,\footnote{These intermediate releases were selected because they enable equally spaced time intervals between new datasets for comprehensive static science analysis: years 1, 4, 7, and 10 corresponding to DR2, DR5, DR8, DR11.} DR5 and DR8, as well as identifying solutions that enable rolling (at the strength recommended in \citetalias{PSTN-055}) while increasing the uniformity of key data releases, a Uniformity Task Force developed alternative rolling implementations. 
 
 {\it Uniform rolling} implements interruptions of rolling before specific data releases to increase the uniformity of those releases and recover an acceptable level of uniformity at key years---see Figs.~\ref{fig:stripiness} and~\ref{fig:uniform-rolling}.  It was found that uniform rolling permits the full survey area to be used for cosmological analysis at years 4 and 7, whereas in previous rolling versions, approximately 35\% of the cosmological constraining power\footnote{Here we quantify cosmological constraining power through emulated forecasts of combined constraints from cosmological weak lensing and large-scale structure measurements \citep{2022ApJS..259...58L}.  The constraints assume a $w_0 w_a$CDM cosmological model, with $w_0$ and $w_a$ entering as two parameters in the dark energy equation of state.  The constraining power is quantified through the area of the uncertainty contours in the $(w_0, w_a)$ part of parameter space, marginalizing over other cosmological parameters and systematic uncertainties -- then taking the inverse of that area (so that higher values mean lower uncertainty, i.e., tighter cosmological constraints).  However, this can be considered more generally as a proxy for how well we are measuring cosmological structure growth, translating into tighter constraints on the amplitude of matter fluctuations if a $\Lambda$CDM cosmological model is assumed.} was lost at those years due to the need for area cuts. 

We note that over the 10-year LSST, the envisioned two-region rolling strategy can be implemented with at most four rolling cycles (that is, starting rolling in Y2 and ending rolling in Y10) where a cycle is defined as a pair of two years, where the high- and low-intensity regions are swapped. Uniform rolling requires limiting rolling to three cycles. Rolling primarily benefits science sensitive to timescales of $\sim24-48$ hours.\footnote{Shorter time scales are primarily covered by observations in triplets, as discussed in \citetalias{PSTN-055}.} 
We note that these time scales had been identified as sensitive and requiring additional improvements in \citetalias{PSTN-055} within the recommendation on rolling (\citetalias{PSTN-055} \S2.4.1): 

\begin{quote}
    {[\citetalias{PSTN-055} \S2.4.1] ``[...] the SCOC recommends the LSST cadence be designed to ensure coverage of time scales in the hours-to-one-day range by carefully tuning survey parameters in combination. Performing three visits per night by default is not recommended, but a combination of preferentially pushing a third visit to the following night [...] and requesting a third visit within a night once every several nights ($\sim 1$ week) would achieve this goal.'' }
\end{quote}

%While not designed for this purpose, t
The implementation of the 0.9 strength, two-sky-areas rolling with four cycles (\baseline{3.2}-\baseline{3.5}) improved coverage at 24-48 hours (see figure \autoref{fig:rolling_sampling}) over the \baseline{3.0} that accompanied \citetalias{PSTN-055}.


\begin{figure}
    \centering
    %\begin{overpic}[width=0.8\textwidth]{figures/rolling_sampling.png}{figures/rolling_nsamples.png}
    %    	\put(50,30){\color{lsstblue}\huge DRAFT}
    %\end{overpic}

    \includegraphics[width=0.4\linewidth]{figures/rolling_sampling.png}\includegraphics[width=0.58\linewidth]{figures/rolling_nsamples.png}
    \caption{The mean number of observations at 24, 48, 72, and 96 hours in all bands, $g$-band only, $r$-band only, and $z$-band only as a function of the number of rolling cycles: each cycle of rolling improves the number of samples by about 7\%. On the left, the ratio of samples normalized to the number of samples when not rolling is shown for two \opsim s with 3 rolling cycles (\texttt{roll\_3\_v3.4\_10yrs} and \texttt{roll\_uniform\_early\_half\_mjdp0\_v3.4\_10yrs}) and two \opsim s with 4 rolling cycles (\baseline{3.3} and \baseline{3.4}). On the right, the absolute number of samples in $g$, $r$, and $i$ is shown. Note the overall small numbers of samples in these time scales when not rolling: $<5$ in $g$ and  $10<ns< 15$ in $r$, $i$, and $z$ at 24 hours. The rolling strategy improves sampling between 24 and 96 hours. Recall that \citetalias{PSTN-055} concluded that the sampling enabled by the rolling implemented in \baseline{3.0} (in three cycles) was still insufficient while an additional cycle improves this sampling by $\sim$5-7\%. %A fourth cycle of rolling improves this sampling by $\sim7\%$.
    }
    \label{fig:rolling_sampling}
\end{figure}

Considering the above input for multiple science cases, the SCOC recognizes the positive impact that this rolling implementation (3-cycle {\it uniform rolling}) has on static cosmological and extragalactic probes and considers this a promising solution for the uniformity concerns raised in \citetalias{PSTN-055}, with limited detrimental impact on time-domain probes. However, this implementation of rolling is a significant and new departure from earlier implementations, and the number of cycles of rolling had not been previously explicitly discussed as a parameter in the survey strategy. Furthermore, the current Uniform Rolling implementation requires rolling to start early in Y2 (in the current implementations, rolling starts on survey day $\leq 400$) but, as discussed in \citetalias{PSTN-055}, rolling shall not start until sufficient sky coverage has been achieved to enable proper photometric calibration.%\footnote{\tbd{A discussion and references for "good enough" calibration should be added, see Eli 2023 PCW - @fed}} 
\begin{figure}
  \centering
    \includegraphics[width=0.75\linewidth]{figures/stripiness_metric.png}
    \caption{
    A quantitative assessment of the non-uniform exposure time variation vs.\ year under different observing strategies.  The test statistic plotted on the vertical axis effectively measures the fractional difference between the variations in depth between the northern and southern Galactic regions, with a value of 0 indicating that the two are the same, as expected for a perfectly uniform survey.  The light green shaded envelope between the dashed black lines indicates the region for which we consider the stripe features to be negligible (meaning manageable within the limits of existing analysis algorithms).   The narrower dark green shaded envelope shows the expected statistical fluctuations for a survey without rolling, as estimated using the \texttt{noroll\_v3.4} strategy simulation.  As shown, at the highlighted years (1, 4, 7, 10), the uniform rolling strategy (\texttt{roll\_uniform\_early\_half\_mjdp0\_v3.4\_10yrs}) is very close to uniform within the level of statistical fluctuations at Y4 and Y7 (DR5 and DR8), while the \baseline{3.4} strategy is highly non-uniform, especially in those years.
}
    \label{fig:stripiness}
\end{figure}
For these reasons, the SCOC is not committing at this time to recommend any specific implementation of rolling, beyond confirming the strength of 0.9 and two-region strategy. Since in all current implementations,
rolling does not begin until Y2, the SCOC intends to continue investigating rolling implementations and their impact throughout Y1, with the support of the community, and release a recommendation of how to implement rolling as part of its first annual recommendation ahead of Y2 of Operations. In particular, we intend to (1) investigate sensitivity to the outcomes of Y1, (2) ensure the community has time to evaluate the potential impacts of these changes that are not currently highlighted by our metrics, and (3) refine the uniform rolling implementation details.

{\bf The SCOC recommends that the time domain community, particularly those interested in phenomena that have evolutionary timescales of hours-to-days, urgently quantify the impact of the proposed uniform rolling compared to rolling in four cycles. For this purpose, while the baseline is implemented with 3-cycle uniform rolling, the Survey Strategy team has prepared \texttt{v3.6} \opsim s  with different rolling implementations.}

{\bf Further, the SCOC restates its recommendation that Data Management scopes a plan for producing uniform data releases in DR5 and DR8, in addition to the standard data releases, and that the cost of the development and storage of these additional data releases is scoped and shared with the scientific community.} Even if produced by the Rubin DM, uniform data releases will require the input of the DESC and extragalactic science community at large to develop the algorithm that will achieve sufficient uniformity. Understanding the cost of producing two additional {\it uniform} data releases is necessary to compare this cost to the scientific cost of three vs. four cycles of rolling, to be measured by the community (see previous paragraph). In addition, if rolling cannot start early enough to interrupt rolling ahead of DR5 and DR8, this remains the only alternative solution currently identified to achieve sufficient uniformity. Sharing information on the cost of additional data releases will place the community in a position to, if needed, advocate for and secure funding for this purpose.

The SCOC is thankful to the Uniformity Task Force, chaired by Rachel Mandelbaum, which provided invaluable contributions and analysis that led us to this recommendation.

%\tbd{ NOTE: we should add direct answers to these questions: Rachel? The community should specify the desired and necessary requirements for uniformity to enable the exploration of data processing solutions to this problem. Depending on the feasibility of a solution to ensure sufficient uniformity, the SCOC recommendation on rolling may be re-evaluated.}  \tbd{From Rachel: There isn't a single number we can give regarding requirements.  Would it be sufficient to say that the uniformity task force has provided quantitative metrics (some heuristics, some truly science-driven) that address this recommendation?}



\begin{figure}
    \centering
    %\begin{overpic}[width=0.8\textwidth]{figures/Rolling.png}
    %    	\put(50,30){\color{lsstblue}\huge DRAFT}
    %\end{overpic}
    \includegraphics[width=0.32\linewidth]{figures/baseline_v3_4_Nvisits_Year_4_HEAL_SkyMap.png}
    %\begin{overpic}[width=0.8\textwidth]{figures/RollingCompare.png}
    %   	\put(50,30){\color{lsstblue}\huge DRAFT}
    %\end{overpic}
    \includegraphics[width=0.32\linewidth]{figures/noroll_v3_4_Nvisits_Year_4_HEAL_SkyMap.png}
    %\begin{overpic}[width=0.8\textwidth]{figures/RollingUniform.png}
    %    	\put(50,30){\color{lsstblue}\huge DRAFT}
    %\end{overpic}
    \includegraphics[width=0.32\linewidth]{figures/roll_uniform_early_half_v3_4_Nvisits_Year_4_HEAL_SkyMap.png}

    \includegraphics[width=0.6\linewidth]{figures/roll_colorbar.png}
    \caption{Comparison of the depth of LSST at the end of Y4 (DR5) under different rolling strategies. The left plot shows the LSST number of visits map for a standard implementation of rolling at strength 0.9 in two sky regions designed as four longitudinal stripes. The center plot represents an implementation of LSST without rolling, for comparison, and provides an upper limit to expected uniformity. The right plot shows the implementation of Uniform Rolling described in this section and implemented in \baseline{4.0}.}
    \label{fig:uniform-rolling}
\end{figure}

\FloatBarrier


\subsection{Galaxy}\label{sec:galaxy}
In \citetalias{PSTN-055} the SCOC identified areas of work needed to finalize the WFD survey strategy on the Galactic sky and special regions of interest to Galaxy science, including the Clouds and South Celestial Pole, which can be observed within the WFD but with different observing choices than the low-dust footprint, of primary interest for extragalactic science. 

\begin{quote}
    {[\citetalias{PSTN-055} \S4] ``The SCOC is not ready to finalize a recommendation for the filter balance in the Galactic Plane, or for a final Galactic Plane/Bulge footprint, or the rolling scheme to be implemented on the Galactic Plane. The SCOC will work with the SMWLV and TVS SCs to ascertain the best solutions for Galactic science regarding filter balance and footprint. These decisions should, however, not impact decisions relating to the WFD and the time spent collectively on Galactic regions should not change. Galactic Plane pencil-beam surveys need to be defined more clearly to assess if they would ultimately result in ``nano-surveys'', which will require a fraction of time too small to be optimized at this stage, or to evaluate the possibility of incorporating them into a final Galactic Footprint recommendation''.}
\end{quote}

This section includes updated recommendations on the Galactic footprint and its observing cadence, including whether rolling should be implemented (\autoref{sec:subG:footprint}), filter balance (\autoref{sec:subG:filterbalance}), and special regions (\autoref{sec:subG:specialregions}).

\subsubsection{Footprint and Time Distribution of Visits}\label{sec:subG:footprint}

Extensive work has already led to the present division of the dense regions of the Galaxy into a high-visit region that encompasses both a large area around the Bulge and a long, thick strip of the Plane, surrounded by a larger area in the Plane with fewer visits. %This division was needed because the overall survey constraints prevent covering this entire area at WFD depth. 
The subsequent efforts of the SCOC and scientific community have been focused on refining these choices.

One feature of the \baseline{3.0} (\citetalias{PSTN-055}) survey in the Plane is that it left a high-visit ``blob'' centered around a Galactic longitude of $l=+45$ surrounded by a lower visit area. This resulted from a previous candidate survey design that included high-visit pencil beams\footnote{In a subset of previous candidate survey designs, ``pencil beams" were a series of 20 high-visit single pointings distributed in galactic longitude with the goal of ensuring the survey sampled a range of stellar environments.} at varying Galactic longitudes along the Plane and considered stellar density, but it was not due to any other specific science goal in this region. Visits centered around this high-declination blob would necessarily have to occur at high airmass, and would additionally be separated from other high-visit areas, reducing survey efficiency.

{\bf The SCOC recommends redistributing the  visits concentrated in the ``blob'' centered around a Galactic longitude of $l=+45$ to cover a low-visit ``barrier'' at $l=+335$ in the Plane and at the border of the Plane and Bulge. This change would give continuous longitude coverage along the Plane from a longitude of $l=+30$ down through $l=+280$ and boost metrics for time-domain science in the Bulge/Plane.}


\begin{figure}
\centering
    \includegraphics[width=0.75\textwidth]{figures/baseline_v3_0_10yrs_nvisits_galactic.png}
    \includegraphics[width=0.75\textwidth]{figures/baseline_v4_0_10yrs_nvisits_galactic.png}
    %\includegraphics[width=0.3\linewidth]{figures/RollingUniform.png}
    \caption{Comparison of the MW footprint (Galactic coordinates) as recommended in \citetalias{PSTN-055} and implemented in \baseline{3.0} through v3.5 simulations and the refined footprint recommended in this document, and implemented in \baseline{3.6} and later.}\label{fig:gpfootprint}
\end{figure}

The SCOC recommends rolling on the low-dust WFD (see \autoref{sec:rolling}), where strips in declination alternate high- and low-intensity monitoring. %Rolling trades detection of transients for improved temporal coverage; for some classes of extragalactic transients, such as SN Ia to be used for cosmology, this tradeoff is strongly positive on net.
However, this rolling implementation need not extend to the dense regions of the Galaxy if it does not provide overall scientific benefits to Galactic science.

In \baseline{3.0}, no rolling is implemented in the Bulge and Plane footprint. The SCOC explored simulations that implemented rolling in both regions or only in the Bulge.
Rolling in both the Bulge and Plane is extremely unfavorable for many Galactic transient metrics, such as microlensing discovery and characterization for a broad range of event timescales, as well as early detection of X-ray binary outbursts.\footnote{The X-ray binary outburst metric is representative of Galactic transients with a typical duration longer than a few days that follow the stellar distribution in the Galaxy, so it has much broader relevance than solely for X-ray binaries.} The outcomes are more complex for rolling in the Bulge alone; while still negative for Galactic transient discovery, rolling in the Bulge has a mixed effect on microlensing metrics. These Bulge simulations particularly aimed to explore whether rolling cadence implementations could boost the early detection and characterization of shorter ($\sim$ few days) timescale microlensing events and anomalies, even if only for a limited survey region. In practice, the improvement was found to be comparatively small, and came at the detriment of the regular, long-baseline monitoring necessary to characterize long-timescale events such as those caused by compact object lenses.   


{\bf The SCOC concluded that rolling on the Galactic footprint would have a net negative effect on the survey as a whole, and recommends no rolling in the Plane or Bulge.}

Finally, the SCOC recommends the redistribution of a small number of Bulge visits to 
a central Bulge field overlapping the planned Roman Bulge survey area, with a goal of more continual monitoring to improve microlensing detection and characterization in this region that will be intensively surveyed by Roman during predetermined seasons.
This recommendation was first implemented in \opsim\ \texttt{roman\_v3.3} and resulted in large improvements for some microlensing metrics, improvements for all microlensing metrics, and no significant negative impact on any other metrics. 

{\bf The SCOC recommends a visit plan consistent with this \texttt{roman\_v3.3} simulation, with the number of redistributed visits to be capped $\sim1600$, as in  \texttt{roman\_v3.3}. However, the timing of the implementation of this augmented observing campaign needs to remain flexible at this time to respond to the yet-to-be-finalized launch date of Roman and the scheduling of its surveys.}


\subsubsection{Galactic Filter Balance}\label{sec:subG:filterbalance}

The filter balance in the Bulge in the current \baseline{3.4} already differs from that used for WFD: the primary difference is fewer visits in $y$, which are redistributed to bluer filters to better optimize Galactic science since $u$ and $g$ are vital for stellar characterization even in the presence of foreground dust. In the WFD, $y$ receives a large number of visits, comparable to the number in $z$ and only slightly less than $r$ or $i$, while $g$ and especially $u$ receive fewer visits. Hence $y$, with its relatively low sensitivity, is the optimal choice for redistribution to bluer bands.

Noting the relatively low sensitivity of the $y$ band, and its resulting negligible reddening advantage over $z$ even in dusty regions, the SCOC considered several simulations that redistributed additional visits in the dense regions of the Galaxy from $y$ to a combination of $z$, $g$, and $u$, while still recognizing the fundamental discovery potential of a multi-filter survey over a broad contiguous area. The main finding from these new simulations was that most existing metrics showed mixed or marginal changes, even where the relative number of visits in $u$ and $g$ substantially increased. The metrics considered included Galactic transients, young stars, detection of several classes of periodic variables, light curve gaps, as well as solar system metrics (since the ecliptic passes through this region).

{\bf The SCOC finds that the adoption of a revised filter balance in the Bulge and Plane with less $y$ and more $z$, $g$, and $u$ compared to the present baseline is potentially beneficial on the net, but that existing metrics are not adequately sensitive to the explored filter balance changes for some expected science cases. The SCOC concludes that a survey using the currently implemented filter balance in the Bulge and Plane in \baseline{3.4} will produce excellent science and the LSST can start with this implementation.} 
However, the SCOC also welcomes input from the community whose science is affected by the details of filter balance in the dense regions of the Galaxy to help define improved metrics that could lead to further optimization in future years.


\subsubsection{The LMC/SMC and South Celestial Pole}\label{sec:subG:specialregions}

The scientific goals of the survey in the region of the LMC and SMC (together MCs) \footnote{There is an effort underway avoid using the current full name of the MCs, as reasoned in \url{https://physics.aps.org/articles/v16/152}.  We adopt the acronyms LMC/SMC without expanding them into the full name here to reflect the broad and inclusive reach of Rubin LSST.} 
and South Celestial Pole (SCP) differ somewhat from WFD. In particular, the major areas of focus of the survey in the main bodies of the LMC and SMC are microlensing and other variable/transient science. In the peripheries of the MCs, including the South Celestial Pole region, the central goal is to detect dwarf satellites and other low-surface-brightness stellar substructures.
These goals are supported in the current baseline, as the MCs are covered with the same number of visits as the WFD, while the SCP region, only observable at relatively high airmass, has a low number of total visits, but sufficient to detect many potential dwarf satellites and substructures. However, the current baseline also adopts the WFD filter baseline in the MC and SCP regions, which may not be ideal for the stated goals.


A number of simulations were considered that used an alternate filter balance for both the Clouds and the South Celestial Pole, moving visits out of $z$/$y$ and toward $u$/$g$ in both regions. These simulations show large improvements in metrics relevant to the detection of low surface-brightness dwarfs as well as some improvements in microlensing and variable star/transient metrics. 

{\bf The SCOC recommends a bluer filter mix in these regions, bounded by the requirement that the increased number of dark-time visits in a relatively narrow range of right ascension does not affect other areas of the survey.} 

\begin{figure}
  \centering
  \includegraphics[width=0.32\linewidth]{figures/baseline_v4_0_10yrs_NVisits_g_band_HEAL_SkyMap.png}\includegraphics[width=0.32\linewidth]{figures/baseline_v4_0_10yrs_NVisits_i_band_HEAL_SkyMap.png}
  \includegraphics[width=0.32\linewidth]{figures/baseline_v4_0_10yrs_NVisits_z_band_HEAL_SkyMap.png}

  \includegraphics[width=0.6\linewidth]{figures/baseline_v4_0_10yrs_NVisits_colorbar.png}
    \caption{The different filter balance in the LMC and SMC regions, compared to the rest of the WFD can be seen when comparing the number of exposures in $g$, $i$, and $z$ at the end of the 10-year survey (the rectangular region near the SCP). These figures also demonstrate the bluer filter
    balance in the Milky Way region.}
    \label{fig:enter-label}
\end{figure}

The SCOC is thankful to the Galaxy Survey Strategy Task Force, chaired by Jay Strader and Rachel Street, which provided invaluable contributions and analysis that led us to this recommendation.





\FloatBarrier





\subsection{Targets of Opportunity (ToO)}\label{sec:ToO}
 
In \citetalias{PSTN-055} the SCOC recommended the implementation of a ToO program that should:

\begin{quote}
    {[\citetalias{PSTN-055} \S2.8] ``[...] be contained to $\leq$3\% of the LSST time. The SCOC recommends that Rubin organizes a workshop in 2023 to bring together members of the scientific community, members of Rubin Observatory (including observing and scheduler specialists, and Data Management specialists), and members of the SCOC to define the details of the implementation of the Rubin ToO program. This workshop should produce a document detailing recommendations for implementation, including suggestions for the questions outlined above, that the experts agree would accomplish the scientific goals of the program.''}
\end{quote}

A meeting was organized in March 2024 (Rubin ToO 2024\footnote{\url{https://lssttooworkshop.github.io/images/Rubin_2024_ToO_workshop_final_report.pdf}}) with the explicit purpose of making a community recommendation for a Rubin ToO program within bounds previously established by the SCOC. After evaluating this community consensus report and considering simulations of its implementation, the SCOC finds that the impact on WFD science is generally small and that the proposed ToO programs have the potential to lead to important scientific results. 

 {\bf The SCOC recommends the implementation of a LSST ToO program as detailed in the community report Rubin ToO 2024: 
Envisioning the Vera C. Rubin Observatory LSST Target of Opportunity program\footnote{\url{https://docs.google.com/document/d/1WE4NGl3dFOVGo7lzpyG1fe_JiX9m-kLl5JYQkhu9iso/edit?usp=sharing}} (hereafter RubinToO2024) by the scientific community at large}. 


The impact of including a ToO program as recommended in RubinToO2024 is shown in \autoref{fig:too}. In the current implementation, the program takes between 3\% and 4\% of the survey time. While this is slightly in excess of the recommendation in \citetalias{PSTN-055}, we are still improving the efficiency of the program's implementation, and the current implementation likely represents an upper limit as no triggered sequence is terminated due to reclassification of the event and/or as the transient is identified.
We note that metrics that are very sensitive to the number of WFD observations collected, like SNIa cosmology and Kilonova discovery MAFs suffer a $\sim5\%$ impact. However, the SCOC holds that the potential for discovery of KN counterparts of MMA triggers, and by the promise of KN counterparts of gravitational waves as cosmological probes \citep[\eg ,][]{PhysRevResearch.2.022006, gianfagna2024potential} compensate for this loss. We further note that the data collected within the ToO program, with a denser cadence and deeper images, can result in an effective dataset for the study of fast transients alternative to the WFD data. A negative impact is also seen in some Solar System metrics, however, the ToO program also enhances the goals of the Solar System community. Also see \autoref{sec:summary} for further discussion and visualizations of the impact of ToOs.

\begin{figure}
  \centering
    \includegraphics[height=0.38\linewidth]{figures/baseline_v36_wfd_static.png}\includegraphics[height=0.38\linewidth]{figures/baseline_v36_wfd_transient.png}
    \caption{The impact of the inclusion of the ToO program on static (left) and transient and variable (right) LSST science. Note that the marginal negative impact on the number of well-characterized SNIa (\texttt{SNIa N} MAF on the right) and identifiable Kilonovae (\texttt{KNe-} MAFs in the same plot) in the WFD is compensated, respectively, by the potential for the discovery of KN counterparts of MMA triggers, and by the promise of KN counterparts of gravitational waves as cosmological probes \citep[\eg ,][]{PhysRevResearch.2.022006, gianfagna2024potential}.}
    \label{fig:too}
\end{figure}
This report identified several different classes of ToOs for which Rubin's observations are well-justified. The vast majority of ToOs will be to follow up gravitational wave (GW) events, while a much smaller number of neutrino and solar system ToOs are expected. 

The current LIGO-Virgo-KAGRA (LVK) GW observing run (Observing Run 4 or O4) will end before the start of LSST. Hence, GW ToOs will not commence until the start of the Observing Run 5 (O5) of the LVK detectors. We note that the start time of O5 has no expected impacts on the LSST WFD or the ToO program. Improved system performance, primarily afforded by the consistent working of three detectors (with similar sensitivity), will maximize the scientific productivity of the Rubin ToO program while reducing the impact on other programs. Two working LIGO detectors at their design sensitivity, combined with a third detector working at 30-50\% that of LIGO, will reduce the skymaps to tractable sizes for rapid Rubin coverage. We encourage the LVK science collaboration and the International Gravitational Wave Network (IGWN), to prioritize a high-performing system with three working detectors over an early start of the O5 run. 

As the GW component of the ToO program takes the largest amount of time and has the most impact on WFD, to enable optimal use of Rubin resources, {\bf the SCOC recommends that a meeting to follow Rubin ToO 2024 be organized closer to the start of O5 to refine the GW follow-up survey strategy with improved knowledge of the expected performance of the GW detector networks and systems in O5 and of the performance of the full Rubin system.}
%

There is no comparable time restriction for the solar system ToO program (to follow up potentially hazardous asteroids) or the neutrino ToO program (to follow-up high-energy neutrinos or those from a Galactic supernova). Hence, 
%{\bf the SCOC recommends that the solar system and neutrino ToOs should start as soon as possible: as soon as suitable templates are available.} 
%KV suggests replacing line above with:
\textbf{ the SCOC recommends that the solar system and neutrino ToOs should start as soon as possible. This would be as soon as suitable templates are available for neutrino ToOs and after enough time to assess both the PHA impactor false positive and event rate with the influx of Rubin discoveries (which RubinToO2024 estimated will take $\sim$3 months).}



For all ToOs, to enable ToO response from the Rubin system, a high level of automation is required. For each potential ToO, a response shall be predetermined algorithmically, including which targets Rubin responds to and the sequence of observations, based on the transient’s characteristics. Informal systems can easily lead to mistakes. For this reason, {\bf the SCOC recommends that Rubin only consider potential ToOs that emanate from vetted discovery and distribution systems that produce and dispatch fully machine-readable alerts.\footnote{At the time of writing, the SCOC understands that full automation is not currently in place for all IceCube neutrino triggers.}} The SCOC considers the current list of vetted systems to be: LIGO-Virgo-KAGRA (gravitational waves); IceCube (neutrinos); SNEWS (neutrinos); JPL Scout or Sentry for potentially hazardous asteroids. The SCOC will evaluate future systems for inclusion in this list (\eg , a new neutrino observatory) on formal request.

Human input may still be required to evaluate in real time the value of a ToO trigger and the specific response.  One (or more) Rubin members %(assigned on any given night from a group of Rubin Scientists) 
should review triggers and be allowed to, if desired, overwrite the algorithmic decision to pursue/not pursue a ToO or interrupt the ToO observing sequence. Further, to ensure that appropriate expertise is available, an Advisory Committee of community members can interact and advise the observer in real-time, with communication initiated either by the committee or by the observer. 
This Advisory Committee should be composed of community members and have a nomination-selection process (including self-nomination) to be outlined in detail before the start of survey operations, ensuring broad coverage of scientific competence and diversity along all relevant axes.
The committee, observers, and Rubin leadership will review the ToO outcomes post factum to advise on program changes. 
{\bf The SCOC recommends real-time human review of potential ToO triggers and the establishment of a Rubin ToO Advisory Committee as described above.} The Advisory Committee should be composed of community members and, collectively, have relevant expertise on all ToO science cases (Solar System, Neutrino, Gravitational Waves, and any science case that may be added to the program in the future). The Advisory Committee should be empowered to propose changes to the observing strategy based on the outcomes of the program and scientific developments. The SCOC will also solicit and consider feedback on the implementation of these ToO programs as necessary to ensure they meet the science goals outlined in the community ToO report. 



\subsection{Snaps}\label{sec:snaps}
%\tbd{The SCOC awaits commissioning assessments of the viability of collecting images in a single 1x30s exposure in all filters (rather than 2x15s), which would lead to an increase in efficiency. The SCOC has thus far seen as favorable a potential switch to a single 1x30s exposure and the associated efficiency gain. If commissioning reveals that a 1x30s exposure is indeed technically viable, the SCOC should review the benefits (and potential drawbacks) of visits in a single exposure and, if adopted, reassess its recommendations in the light of this increased efficiency.}

While the LSST has originally been designed to collect two 15-second snaps for each visit, primarily to remove cosmic rays, there is an opportunity to move to collect a single 30-second exposure\footnote{Note that with the recommendation on $u$-band exposure length and filter balance (\autoref{sec:filterbalance}), the exposure time in $u$ is 38 seconds while the exposure time in all bands but $u$ drops to $\sim$29 seconds (simulations show the resulting exposure time in $grizy$ to be 29.2 seconds). However, for convenience, we will continue to refer to a ``2x15 seconds'' and ``1x30 seconds'' implementation.}, as it appears that cosmic rays can be reliably rejected from a single image. The feasibility of this plan remains to be ascertained in commissioning (including from on-sky images). However, the SCOC has conducted a science-driven analysis of this proposal. 

Based on simulations, going from 2x15s to a single 30s exposure brings a gain in efficiency equivalent to ~7-9\% of the survey time (as a result of reduced camera read-out time). 


Saturation limits will be slightly higher but this will only impact a small number of objects compared to the large volume of sources in the LSST universe. Other surveys are better equipped to work with those targets that are bright for LSST.
Some science cases (\eg\ Cataclysmic Variables and Flares, very fast moving solar system objects) could benefit from the separate exposures, but the planned data processing for the individual snap images is more limited than that applied to the combined visit, so these science cases need to rely on pipelines contributed by the community and user-generated data products. Additionally, for these cases too, other surveys are better equipped to work within those time scales.

Thus, the SCOC does not see scientific opportunities associated with retaining the two 15s snaps that can compete with the 7-9\% gain in survey efficiency.

{\bf The SCOC recommends that, if the technical feasibility is confirmed in commissioning, the survey be conducted with single exposures. With our recommendation of modifying the exposure time for $u$ band to 38 seconds, and compensating for this extra $u$ band survey time by a short decrease in exposure across all other bands, the single visits would be $\sim$1x29 seconds.}\footnote{29.2 seconds from simulations.}

The time gained by avoiding snaps will not be allocated to any specific program in Y1 as the performance of the system is still uncertain. In the future, the SCOC will consider how the additional time may be allocated, including to special programs (\eg, microsurveys), DDFs, WFD, etc., either to compensate for unexpected performance loss or to increase science throughput.

\subsection{Deep Drilling Fields (DDF)}\label{sec:DDF}
%\tbd{The SCOC will continue working in 2023 with the community to identify the specific intra-night cadence that maximizes the science throughput of the DDF survey, while not impacting the science performed by other surveys.}

A general plan for the LSST Deep Drilling Fields (DDF) has been developed over the course of the past 15 years, starting with \cite{2009arXiv0912.0201L} through many further developments and recommendations \citep[\eg ,][]{Brandt:2018,Scolnic:2018,Yu:2020,Kovacevic:2022,Czerny:2023,Zhang:2023,Gris:2023,Gris:2024,PozoNunez:2024}.

The DDF program will include
five DDF pointings. The SCOC recommended in \citetalias{PSTN-055} that  
6-7\% of overall survey time be dedicated to the DDF program, and that each DDF receives $\sim$20k visits
except for the COSMOS field, which should receive $\sim$40k (with accelerated coverage so that COSMOS reaches $\sim$20k visits by the end of Y3). The Euclid Deep Field South (EDFS) has a wider area equivalent to two separate pointings (sharing $\sim$20k visits across the two pointings). \citetalias{PSTN-055} stated: 
\begin{quote}
    {[\citetalias{PSTN-055} \S4] ``The SCOC will continue working in 2023 with the community to identify the specific intra-night cadence that maximizes the science throughput of the DDF survey, while not impacting the science performed by other surveys.''}
\end{quote}


The implementation of DDF intranight visits is still under development. Trade-offs between nightly depth, cadence, season length, and filter balance are still being explored.

{\bf The SCOC that the baseline survey strategy accommodate varying the nightly depth, filters, or cadence of different DDFs throughout the course of LSST, while maintaining the \citetalias{PSTN-055} recommendations for the 10-year depth of each field (including the enhanced COSMOS observations to reach 10-year depth in the first three years).}

Adding this flexibility to DDF observations allows for periods of higher cadence necessary for some transient science \citep[\eg, AGN or supernovae;][]{Yu:2020,Kovacevic:2022,Czerny:2023,PozoNunez:2024,Gris:2023,Gris:2024} and enables more opportunities for concentrated, contemporaneous observations with other surveys (\eg , Euclid, Roman) while maintaining the overall co-added depth for static science.

The SCOC can make some further recommendations on the DDF beyond intranight cadence based on input from the SCOC DDF Task Force.

{\bf The SCOC recommends that DDF observations be sequences of multiple WFD-like visits (as opposed to increased exposure times) to allow rapid alert generation.} 

Retaining sequences of visits in multiple filters within a night allows for deeper per-night measurements through co-adds, while still probing sub-minute timescale sampling over the observation. This approach engages the alert generation infrastructure just like the main WFD survey and also benefits cross-calibration of DDF and WFD observations. 

{\bf The SCOC recommends that the baseline translational dithering scale of DDF observations be reduced from 0.7 degrees to 0.2 degrees (with exploration of even smaller translational dithers compatible with instrumental signature removal and calibration needs).
}

Smaller translational dithers allow DDFs to reach increased co-added depth for static science and increased temporal coverage for time-domain sources. While a larger dither is favored for low-surface brightness science, no nearby clusters or other large, low-surface brightness structures of interest (e.g., nearby galaxies) are included, by design, in the DDF pointings.



{\bf The SCOC urges the Data Management and Alert Production teams to assess the feasibility of, and resources needed for, enabling nightly co-adds of sequential DDF visits and recommends that a path be developed to enable the creation of these co-adds, subtraction with deep templates, and faint alert generation (with higher latency as needed, \eg , after sunrise). 
}

Nightly co-adds are required to take advantage of the increased DDF depth in the time domain. Alerts from nightly co-adds are essential for faint time-domain sources (\eg , high-redshift AGN or supernovae). Longer timescale co-adds (\eg , weekly, monthly, yearly) and alerts should also be considered.





\FloatBarrier

\subsection{Early Science}\label{sec:early}
% \tbd{The SCOC recommends implementing a detailed coordination plan with the Early Science Rubin team to reach a final recommendation on the strategy to be implemented in the first year of the survey, including a scheme for the construction of templates.}

 The SCOC emphasizes that the priority in Y1 of operations should be obtaining a dataset that supports and facilitates science throughout the survey. This includes a dataset sufficient for calibration across the $\sim 20,000$ square degrees of the WFD, including images at different airmasses, illuminations, field crowdedness, etc. 
 
 The SCOC supports Rubin's commitment to acquiring incremental templates throughout Y1 to begin dispatching alerts (via the Alert Brokers) and encourages the Observatory to release alerts as early as possible. The SCOC reviewed the Alert Production team's proposal to prioritize timeliness over the quality of templates and build templates from fewer images ($\geq 3$) in Y1 than in subsequent years. Releasing some alerts in Y1 is an important goal to enable the time domain and solar system science communities to prepare for the full-volume, full-fidelity alert streams to come in subsequent years, as well as increasing the discovery potential of LSST in early operations. Earlier template generation is particularly important for testing solar system alert streams that require post-discovery re-detections of solar system objects.
 However, this goal should not overwrite the priority of obtaining a fully calibrated system by the end of Y1.

 {\bf The SCOC recommends that the filter balance is adjusted as needed in Y1 to acquire a sufficient number of $u$ band images for calibration (and template construction). }

 {\bf The SCOC does not recommend beginning rolling before the end of Y1 to ensure sufficiently uniform sky coverage for cosmological analysis (the DESC expects its first data analysis to be based on DR2), acquire sufficiently good data for sky calibration, and collect a 
 sufficiently complete set of templates across the sky.} In the months following the release of this recommendation, the SCOC will continue to work on the implementation of rolling (\autoref{sec:rolling}) to better understand its interplay with potential Y1 outcomes. 

%\subsection{Euclid}\label{sec:Euclid}
%\tbd{The SCOC shall work in coordination not only with the scientific community but also with the leadership of Rubin and the Euclid mission to identify cadence requirements, co-observing strategies, and paths to produce the data products that will enhance science through the coordinated observing of the EDFS.}

\FloatBarrier




\section{Additional Recommendations}\label{sec:additional}

{\bf A small change to the southern portion of the footprint improves overlap with the Euclid footprint} (see Fig.~\ref{fig:euclid-overlap}) and causes negligible changes in science metrics. 
\begin{figure}
\centering
\includegraphics[width=0.45\textwidth]{figures/baseline_v3_0_10yrs_euclid_overlap.png}
\includegraphics[width=0.45\textwidth]{figures/baseline_v4_0_10yrs_euclid_overlap.png}
%\includegraphics[width=0.45\textwidth]{figures/3.0_south.png}\includegraphics[width=0.45\textwidth]{figures/3.1_south.png}

\caption{Small changes to the southern portion of the footprint improve the overlap with Euclid. \label{fig:euclid-overlap}}
\end{figure}

{\bf The airmass limits for the Near-Sun Twilight microsurvey, introduced with baseline v3.0, were increased from $X=2.5$ to $X=3.0$} in \texttt{v3.2}, corresponding to decreasing the minimum solar elongation reached for this microsurvey from 40 degrees to 35 degrees (the range of solar elongations changed from 40 to 60 degrees in \texttt{v3.0} to 35 to 47 degrees in\texttt{v4.0}). This improves the likelihood of discovery of objects with interior-to-Earth orbits, increasing the survey sensitivity to this niche of discovery space. The recovered population of objects
interior to Venus at magnitude $H\leq20$ goes from $\sim4\%$ to $\sim40\%$ in \texttt{v3.2} and later. The impacts outside the microsurvey are negligible.

\clearpage

\section{Additional changes introduced throughout the \texttt{v3.x} \opsim s } \label{sec:opsimchanges}

%Starting with \texttt{v3.6} some important assumptions underlying the simulations were updated: 

\begin{itemize}
\item As of \baseline{3.6}, the downtime in year one was increased to reflect a more realistic transition into operations. This change adds approximately eight weeks of downtime reducing the number of visits by $\sim$5\%. The downtime in Y1 is simulated to be maximal early on and decreased to the level expected for the general LSST survey by the end of the first year (\autoref{fig:downtime}). Future simulations will aim to improve the unscheduled downtime model to better align with expectations from the Rubin Observatory Operations team. 

\item As of \baseline{3.6}, the effect of jerk on slew time is included in the simulations, and thus included in scheduling choices. Functionally, this slightly increases the overhead and decreases survey efficiency (\autoref{fig:downtime}).

\begin{figure}[!h]
    \centering
    \includegraphics[width=0.37\linewidth]{figures/downtime_v3_5_year1.png}
    \includegraphics[width=0.37\linewidth]{figures/downtime_v4_0_year1.png}
    \includegraphics[width=0.24\linewidth]{figures/downtime_v4_0_year1_legend.png}
    \caption{The time within each night of LSST in Y1 divided into on-sky exposure time, overhead for those exposures (shutter and readout time), time spent slewing, and downtime due to weather, scheduled maintenance activities, or unscheduled engineering. Before \baseline{3.6} (left plot), simulations only included steady-state expected engineering downtime, modeled as full-night downtime blocks. The \baseline{4.0} simulation (right plot) includes additional unscheduled downtime time within the first 380 nights of the survey, including breaks as short as an hour to reflect the need for engineering early in the survey. }
    \label{fig:downtime}
\end{figure}


\item The baseline simulations that accompanied the previous SCOC report (\citetalias{PSTN-055}) had a start time of October 1, 2023. As of \baseline{3.2}, the start date of the survey was updated to  May 1, 2025 to match the Project forecast at that time. %The \baseline{4.0} also has a start date of May 1, 2025. 
Future simulations will be updated to match the LSST forecasts.\footnote{\url{ls.st/dates}}. In \texttt{v3.4} we began investigating the effect of changing the start date of the survey. The timing of the start of the survey has an impact on various transient and variable  metrics; the performance changes observed in \autoref{fig:start_dates} are primarily random in nature and generally reflect stochasticity in the metrics themselves, but may also be a result of the interplay between observable sky, rolling schedule, and seasonal weather patterns. The effects are generally comparable with uncertainty associated with weather, as \autoref{fig:startdates_weather} shows.
\end{itemize}

%\begin{figure}
%    \centering
    %\includegraphics[width=0.75\linewidth]{figures/baseline_v4_0_10yrs_Hourglass_year_0-1_HOUR_Hourglass.pdf}
  %  \caption{This plot shows filter changes per observing night. Downtime (due to weather or telescope scheduled and unscheduled maintenance) is marked by white blocks. Simulations starting with \baseline{3.6} include updated Y1 downtime schedules to reflect realistic expectations of higher downtime periods at the start of the survey. Compared to \baseline{3.5}, \baseline{3.6} and later have approximately eight additional weeks of unscheduled downtime. The impact of the additional downtime is discussed in \autoref{sec:summary}.}
%    \label{fig:enter-label}
%\end{figure}


\begin{figure}
    \centering
    \includegraphics[width=0.9\linewidth]{figures/start_date_scoc_heatmap.png}
    \caption{Performance of a subset of key metrics for implementations of \baseline{3.4} with different start dates, offset by ``offset'' days from the May 1, 2025. Seasonal weather patterns interact with scheduler choices and the timing of rolling. Static science metrics are generally unchanged or only marginally affected, while transient and variable science sees larger impacts. }
    \label{fig:start_dates}
\end{figure}


\begin{figure}
    \centering
    \includegraphics[width=0.85\linewidth]{figures/uncertainties_v3_4.png}
    \caption{With each baseline, we produce a series of simulations with varying start dates and weather patterns. The uncertainty in metric values associated with variations in weather-related downtime is similar to the uncertainty observed changing the start date of the survey, demonstrating that the effect of start date changes is dominated by weather-related effects.}
    \label{fig:startdates_weather}
\end{figure}
\clearpage
\section{Summary of SCOC Phase 3 recommendations}\label{sec:summary}

The following list summarizes the Phase 3 recommendations for the LSST survey strategy which, together with the recommendations in \citetalias{PSTN-053} and \citetalias{PSTN-055}, define the 10-year LSST survey strategy starting plan.   The recommendations included in this report are listed below.

\begin{enumerate}
\renewcommand{\labelenumi}{\roman{enumi}.} 
%[i/]

\item The SCOC recommends swapping $u$- and $y$-band according to the moon phase. Having the $z$ filter always available produces benefits for SN cosmology while preserving coverage on short timescales. This recommendation is implemented starting in \baseline{3.2} (\autoref{sec:filterswap}).

\item All three mirrors in the system will now be coated in silver, which increases throughput in all bands bluer than $u$ but decreases $u$ throughput (by $\lesssim30\%$). Hence, the SCOC recommends an increase of the exposure time in $u$-band to 38 seconds per visit and an increase of the number of $u$-band visits of 10\% compared to \baseline{3.0}. To compensate for the additional time dedicated to $u$-band the SCOC recommends decreasing the exposure time in the other bands. Simulations show that this corresponds to a decrease of 0.8 second exposure time per visit in all other bands (\autoref{sec:filterbalance}).

\item {The SCOC confirms the recommendation for rolling in two sky areas at 0.9 strength on the WFD low-dust footprint (\citetalias{PSTN-055} \S2.5). We are adopting the rolling uniform strategy designed by the Uniformity Task Force in \baseline{3.6} simulations, which implements three cycles of rolling, but we will continue to investigate implementations of rolling until our Y1 recommendation because rolling will not begin before the start of Y2 with any of the implementations under consideration.  The SCOC recommends that the time domain community, particularly those interested in phenomena that have evolutionary time scales of hours-to-days, urgently quantify the impact of the proposed uniform rolling compared to rolling in four cycles. Further, the SCOC restates the recommendation that Data Management scopes a plan for producing uniform data releases in DR5 and DR8, in addition to the standard data releases, and that the cost of the development and storage of these additional data releases is scoped and shared with the scientific community (\autoref{sec:rolling}).} 

\item The SCOC concludes that rolling on the Galactic footprint would have a net negative effect on the survey as a whole, and recommends no rolling in the Plane or Bulge (\autoref{sec:subG:footprint}).


\item The SCOC recommends redistributing the visits concentrated in the ``blob'' centered around a Galactic longitude of $l=+45$ (see \autoref{fig:gpfootprint}, top) to cover a low-visit ``barrier'' at $l=+335$ in the Plane and at the border of the Plane and Bulge. This change would give continuous longitude coverage along the Plane from a longitude of $l=+30$  through $l=+280$ and boost metrics for time-domain science in the Bulge/Plane (\autoref{sec:subG:footprint}).

\item The SCOC recommends a visit plan consistent with the \texttt{roman\_v3.3} simulation, capping the number of visits redistributed to the Roman Bulge field at the value (\mbox{$\sim$1,600}) used in this simulation. However, the timing of the implementation of this augmented observing campaign should remain flexible to respond to the final Roman launch date and survey scheduling (\autoref{sec:subG:footprint}).

\item The SCOC finds that the adoption of a revised filter balance in the Bulge and Plane with less $y$ and more $z$, $g$, and $u$ compared to the present baseline is potentially beneficial for a broad range of science, but that existing metrics are not adequately sensitive to the explored filter balance changes for some expected science cases. The SCOC concludes that a survey using the currently implemented filter balance in the Bulge and Plane in \baseline{3.4} will produce excellent science and the LSST can start with this implementation (\autoref{sec:subG:filterbalance}).

\item The SCOC recommends a bluer filter mix in the SMC, LMC, and SCP regions, bounded by the requirement that the increased number of dark-time visits in a relatively narrow range of right ascension does not affect other parts of the survey (\autoref{sec:subG:specialregions}).

\item The SCOC recommends the implementation of an LSST ToO program as detailed in \emph{Rubin ToO 2024: 
Envisioning the Vera C. Rubin Observatory LSST Target of Opportunity program
}\footnote{\url{https://docs.google.com/document/d/1WE4NGl3dFOVGo7lzpyG1fe_JiX9m-kLl5JYQkhu9iso/edit?usp=sharing}} by the scientific community at large (\autoref{sec:ToO}).

\item The SCOC recommends that a meeting to follow Rubin ToO 2024 is organized closer to the start of the LKV O5 run (expected 2027) to refine the GW follow-up survey strategy with improved knowledge of the expected performance of the GW detector networks and systems in O5 and of the performance of the full Rubin system. 

\item The SCOC recommends that Solar System and neutrino ToOs start as soon as possible: when templates are available and event rates and false positive rates are appropriately assessed. (\autoref{sec:ToO}).

\item The SCOC recommends that Rubin only consider potential ToOs that emanate from vetted discovery and distribution systems that produce and dispatch fully machine-readable alerts (\autoref{sec:ToO}).

\item The SCOC recommends real-time human review of potential ToO triggers and the establishment of a Rubin ToO Advisory Committee as described above (\autoref{sec:ToO}).

\item The SCOC recommends that, if the technical feasibility is confirmed in commissioning, the survey is conducted with single exposures. With our recommendation of modifying the exposure time for $u$-band to 38 seconds, and compensating for this extra $u$-band survey time by short decrease in exposure across all other bands, the single visits would be \mbox{$\sim$1$\times$29} seconds (\autoref{sec:snaps}).

\item The SCOC recommends that DDF observations should be sequences of multiple WFD-like visits (as opposed to increased exposure times) to allow rapid alert generation (\autoref{sec:DDF}).

\item The SCOC recommends that the baseline translational dithering scale of DDF observations should be reduced from 0.7 degrees to 0.2 degrees (with exploration of even smaller translational dithers compatible with instrumental signature removal and calibration needs). (\autoref{sec:DDF}).

\item The SCOC recommends that the baseline survey strategy should accommodate varying the nightly depth, filters, or cadence of different DDFs throughout the course of LSST, while maintaining the Phase 2 (\citetalias{PSTN-055} \S2.6) recommendations for the 10-year depth of each field (including the enhanced COSMOS observations to reach 10-year depth in the first 3 years) (\autoref{sec:DDF}).

\item The SCOC urges the Data Management and Alert Production teams to assess the feasibility of, and resources needed for, enabling nightly co-adds of sequential DDF visits and recommends that a path is developed to enable the creation of these co-adds, subtraction with deep templates, and faint alert generation (with higher latency as needed, \eg , after sunrise) (\autoref{sec:DDF}). 

\item The SCOC recommends that the filter balance is adjusted as needed in Y1 to acquire a sufficient number of $u$-band images for calibration (and template construction) (\autoref{sec:early}). 

\item The SCOC does not recommend beginning rolling before the end of Y1 to ensure sufficiently uniform sky coverage for cosmological analysis (the DESC expects its first data analysis to be based on DR2), acquire sufficiently good data for sky calibration, and collect a 
 sufficiently complete set of templates across the sky (\autoref{sec:early}).

\item The SCOC recommends that the airmass limit for the Near-Sun Twilight microsurvey is increased to $X=3.0$ (\autoref{sec:additional}).

\item The SCOC recommends a slight modification of the \baseline{3.0} footprint to improve overlap with the Euclid footprint (\autoref{sec:additional}). 

\end{enumerate}

These recommendations are implemented in the \baseline{4.0} simulations (\autoref{sec:baseline4_0}). A set of simulations tagged \texttt{v3.6} was made available for the community in early September to assess the impact of different aspects of the recommendation. Note that all of these simulations include the updated, more realistic downtime and effects of slew jerk. In what follows we describe only key simulations; for a more comprehensive description of all \opsim\ s released by the Observing Strategy team, please see posts within the Survey Strategy topic on the LSST Community forum.\footnote{\url{https://community.lsst.org/c/sci/survey-strategy/37}.}

 \autoref{fig:nvisits} shows the number of visits to the WFD survey across simulations, starting with the early vision for LSST (2018) through today's recommendation. \autoref{fig:summary} shows the performance of the survey strategy on a set of core LSST science (top 19 rows) and system metrics (bottom 3 rows) over the same \opsim s. 
 
 First, note that nearly all science cases have seen improvements over time, with some science cases improving by over 50\% (the stretch of the color scale) demonstrating the success of the community-driven approach to survey strategy design that Rubin Observatory has committed to for the past decade, which makes Rubin LSST a much more complete and comprehensively transformational survey. Significant improvements were obtained on most metrics through v3.0 (\citetalias{PSTN-055}). Those are to be attributed to changes of the survey strategy through community input and SCOC recommendations. 
 
 The visible improvement on nearly all metrics between \baseline{3.2} and \baseline{3.3} is attributed to the updated filter transmission curves. The survey strategy is largely unchanged between \baseline{3.3} and \baseline{3.4}; the small changes in performance are to be attributed to  \texttt{rubin\_scheduler} code updates\footnote{See \url{https://survey-strategy.lsst.io/baseline/changes.html}} (also remember that metrics have different degrees of stochasticity in their design). 
 
 The \baseline{3.5} \opsim\ (labeled \texttt{v3.5} in \autoref{fig:summary}) represents an early implementation of the current SCOC recommendations: it includes the new filter balance ($u_{exp}=38$ seconds, $N_u \leq1.1\times ns$, \autoref{sec:filterbalance}), updated Galactic Plane footprint, a slightly extended fraction of time spent on DDFs (still within 7\% as recommended in \citetalias{PSTN-055} and \autoref{sec:DDF} in this document), 3-cycle uniform rolling (but note that, while implemented, the SCOC has not committed to this recommendation, \autoref{sec:rolling}), but does not include snaps or ToOs. Most metrics are stable or improved, except for some time domain metrics (\eg, KNe metrics and SNIa) due to the rolling in three, instead of four cycles and to the new filter balance. 
 
 The overall apparent drop in performance between \baseline{3.5} and \baseline{3.6} is primarily due to the inclusion of slew time jerk effects and more realistic estimates of downtime in Y1 (\autoref{sec:opsimchanges}). \autoref{fig:nvisits} shows the associated drop in number of visits. The \baseline{3.6} has three rolling cycles and includes the ToO program. We provide an \opsim\ consistent with \baseline{3.6}, but without the ToO program to allow the community to investigate the effects that the introduction of ToOs has on LSST. 
 
 The \baseline{4.0} represents the current recommendation outlined in this document. An implementation of this recommendation with four cycles of rolling is offered in \texttt{four cycles v4.0} to enable the investigations of different rolling implementations. Finally, we provide an implementation of \baseline{4.0} with single exposure visits (instead of 2x15 second snaps, \autoref{sec:snaps}) which, pending commissioning outcomes, is the expected observing mode. In this \opsim, the survey time gained by dropping snaps (decreased readtime per visit) is allocated evenly across all observing modes: this includes the WFD, NES, SCP, and Galactic Plane. In the future, with better knowledge of the system as built, the SCOC will consider how the additional time may be allocated, including allocations to special programs,  to compensate for unexpected performance loss or to increase science throughput.
\begin{figure}
    \centering
    %\begin{overpic}[width=0.8\textwidth]{figures/total_nvisits.png}
    %    	\put(50,30){\color{lsstblue}\huge DRAFT}
    %\end{overpic}
    \includegraphics[width=0.9\linewidth]{figures/total_nvisits.png}
    \caption{Number of LSST visits for \opsim s starting with the initial LSST survey strategy proposals (2018, \texttt{v1.X}) thought the current recommendation (this document). The number of observations decreases between \texttt{v3.5} and \baseline{3.6} due to the inclusion of more realistic downtime in Y1, the effects of jerk on slew and scheduling (both discussed in \autoref{sec:opsimchanges}), and the inclusion of the ToO program (\autoref{sec:ToO}). This drop in the number of visit (\mbox{$\sim$5\%}) results in a general loss of performance on all metrics, as seen in \autoref{fig:summary}. The increase in efficiency associated with moving to single-exposure visits (the \opsim\ labeled as \texttt{one snap v4.0}) largely recovers the visits lost between \baseline{3.5} and \baseline{3.6}. Nonetheless, after Y1, when the system performance is better understood, the SCOC will consider how the additional time may be allocated (\autoref{sec:snaps}). }
    \label{fig:nvisits}
\end{figure}

\begin{figure}
    \centering
    %\begin{overpic}[width=0.8\textwidth]{figures/heatmap_3_6.png}
    %    	\put(50,50){\color{lsstblue}\huge DRAFT}
    %\end{overpic}
    \includegraphics[width=0.9\linewidth]{figures/scoc_heatmap.png}
    \caption{LSST performance on key metrics for different \opsim s including the Phase 3 recommendation (this document). From left to right, the sequence shows progressively newer \opsim s, starting with the baseline in effect through 2018 (\texttt{v1.x}) up to the baseline proposed with this recommendation (\baseline{4.0}) and some variations on the latter. Note the stretch of the colorbar: some science cases saw improvements in excess of 50\% from the initial vision for LSST. Notably, starting with \baseline{3.6}, we include more realistic downtime expectations in Y1 and slew jerk. This results in an overall decrease in the number of visits (see also \autoref{fig:nvisits}) with a drop in most metrics. The \baseline{3.6} is also the first baseline which includes the ToO program. A version of \baseline{3.6} without ToOs (\texttt{no\_too v3.6}) is included to allow the community to examine the impact of ToOs. Rolling is implemented in three cycles as of \baseline{3.6} to achieves desired uniformity in DR5 and DR8 (see \autoref{sec:rolling}). The \baseline{4.0} (marked by vertical bars) reflects the recommendations in this document. \texttt{four cycle v4.0} implements the same strategy except for rolling in four cycles, which improves coverage at short time scales.  The SCOC will not commit to a recommendation on the specific rolling implementation until the release of its recommendation during Y1. The community is encouraged to explore the impact of adopting either rolling strategy on their science and share their feedback.  We include an implementation of \baseline{4.0} with visits conducted as single exposures instead of two snaps (\texttt{one~snap~v4.0}). The SCOC recommends the implementation of LSST in single visits, as shown in this simulation; this recommendation is, however, pending commissioning outcomes.}
    \label{fig:summary}
\end{figure}



\FloatBarrier

%\section{Areas of focus leading to and through Year 1}\label{sec:next}.
%\section{The SCOC process during operations}\label{sec:process} 


\appendix
% Include all the relevant bib files.
% https://lsst-texmf.lsst.io/lsstdoc.html#bibliographies

%%%% END OF SECTIONS %%%%%%

\section{References} \label{sec:bib}
\renewcommand{\refname}{} % Suppress default Bibliography section
\bibliography{local,lsst,lsst-dm,refs_ads,refs,books}

% Make sure lsst-texmf/bin/generateAcronyms.py is in your path
\section{Acronyms} \label{sec:acronyms}
\addtocounter{table}{-1}
\begin{longtable}{p{0.145\textwidth}p{0.8\textwidth}}\hline
\textbf{Acronym} & \textbf{Description}  \\\hline

AGN & Active Galactic Nuclei \\\hline
B & Byte (8 bit) \\\hline
CDM & (Lamda) Cold Dark Matter \\\hline
COSMOS & Cosmic Evolution Survey \\\hline
DDF & Deep Drilling Field \\\hline
DESC & Dark Energy Science Collaboration \\\hline
DM & Data Management \\\hline
DR11 & Data Release 11 \\\hline
DR2 & Data Release 2 \\\hline
EDFS & Euclid Deep Field South \\\hline
GW & Gravitational Wave \\\hline
IGWN & International Gravitational Wave Network \\\hline
ITL & Imaging Technology Laboratory (UA) \\\hline
JPL & Jet Propulsion Laboratory (DE ephemerides) \\\hline
KAGRA & Kamioka Gravitational Wave Detector \\\hline
KN & Kilonova \\\hline
LIGO & Laser Interferometer Gravitational-Wave Observatory \\\hline
LMC & Large Magellanic Cloud \\\hline
LPM & LSST Project Management (Document Handle) \\\hline
LSST & Legacy Survey of Space and Time (formerly Large Synoptic Survey Telescope) \\\hline
LVK & LIGO-Virgo-KAGRA \\\hline
M1 & primary mirror \\\hline
M2 & Secondary Mirror \\\hline
M3 & tertiary mirror \\\hline
MAF & Metric Analysis Framework \\\hline
MC & Monte-Carlo (simulation/process) \\\hline
MMA & Multi Messenger Astronomy \\\hline
MW & Milky Way \\\hline
NES & North Ecliptic Spur \\\hline
PCW & Project Community Workshop \\\hline
PST & Project Science Team \\\hline
PSTN & Project Science Technical Note \\\hline
RA & Risk Assessment \\\hline
SC & Science Collaboration \\\hline
SCOC & Survey Cadence Optimization Committee \\\hline
SCP & South Celestial Pool \\\hline
SMC & Small Magellanic Cloud \\\hline
SMWLV & Stars \\\hline
SN & SuperNovae \\\hline
SNEWS & SuperNova Early Warning System \\\hline
SRD & LSST Science Requirements; LPM-17 \\\hline
TVS & Transients and Variable Stars Science Collaboration \\\hline
ToO & Target of Opportunity \\\hline
WFD & Wide Fast Deep \\\hline
YSO & Young Stellar Object \\\hline
photo-z & photometric redshift \\\hline
\end{longtable}

% If you want glossary uncomment below -- comment out the two lines above
%\printglossaries





\end{document}
