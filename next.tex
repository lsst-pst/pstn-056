\section{The future activities of the SCOC and areas of focus leading to and through Year 1}\label{sec:next}

The SCOC is a standing committee of Rubin Observatory, and it will continue its activities through the 10-year duration of LSST. The SCOC will review the survey and issue annual recommendations to the Observatory Director for modifications of the survey strategy in light of scientific outcomes, technical challenges and upgrades, and the evolving scientific landscape. The SCOC will continue to liaise with the scientific community with {\bf annual workshops, liaisons to the SCOC, office hours, and by making its activities public via posts on the LSST Community forum in the Survey Strategy topic}
\footnote{\url{https://community.lsst.org/c/sci/survey-strategy/37}}. As discussed in \autoref{sec:ToO}, the ToO program will be further supported by an Advisory Committee composed of community members that will evaluate the ToO program on an ongoing basis. 

The focus of the SCOC leading up to and into Year 1 will be to review the recommendations shared in \citetalias{PSTN-053}, \citetalias{PSTN-055}, and \citetalias{PSTN-056} in the light of commissioning outcomes and to strategize effective plans for Y1 data collection, including templates, and integrating the data that will be collected before the start of LSST into its recommendations. The deliberations on this topic will necessarily be fluid and evolve rapidly as the commissioning and science verification phases of LSST progress. The SCOC continues to solicit recommendations from the community about scientific prioritization in the collection of templates while restating that the priority in Y1 of operations should be obtaining a dataset that supports and facilitates science throughout the survey. In practice, this means collecting a dataset sufficient for calibration across the \mbox{$\sim$20,000} square degrees of the WFD, including images at different airmasses, illuminations, field crowdedness, etc (\autoref{sec:additional}).

As discussed in \autoref{sec:rolling}, the SCOC will continue to study the impact of the new rolling implementations and of the number of rolling cycles on time-domain science, uniformity of coadds for cosmology and extragalactic science, and all LSST science in general. These investigations will include considerations on the outcomes of Y1 itself as the first year is underway and on the results of the Data Management assessments of feasibility and cost of adding uniform data releases at key years for cosmological analysis (DR5 and DR8). We expect to release a recommendation in the second half of Y1 (likely \mbox{$\sim$2} months before the start of Y2) including a recommendation on rolling implementations.

As discussed in \citetalias{PSTN-055} 
\begin{quote}
    {[\citetalias{PSTN-055} \S4] The SCOC recommends that two microsurveys be scheduled in Y1: the near-sun NEO twilight survey and, if time is available, the Northern Strip survey. Additional microsurveys should be added in the future, when the system characteristics and survey efficiency are better assessed, and a process is recommended to receive and review refined and additional microsurvey proposals after the beginning of LSST.}
\end{quote}

The community is best placed to write effective proposals for  nano- and micro-microsurveys (<0.3\% and between 3\% and 0.3\% of the LSST observing time respectively), and the SCOC best placed to evaluate them, when the capabilities of the system-as-built are estimated with on-sky data. That is, we expect proposals for nano- and micro-survey will be more compelling and realistic after DR1. As a reminder, DR1 will include the first six-months of LSST data, and it will be released \mbox{$\sim$1} year after the start of the survey. An opportunity will be provided to propose timely nano- and micro-surveys before the end of Y1 for science cases of justifiable urgency that require data collected in Y2. This proposal call is expected to be issued no earlier than six months after the start of LSST (when the data for DR1 are collected) with a likely deadline of nine months from the start of LSST; this will allow the SCOC to review and possibly recommend proposals for Y2 implementation. Proposals for nano- and micro-surveys will continue to be solicited and reviewed through the LSST on a regular basis. 

In the past several years, the SCOC has received and considered feedback in the form of community and Science Collaboration reports, communications with the liaisons to the Science Collaborations, discussions held in the SCOC Office Hours, and at the annual SCOC workshops. Feedback from the community will always be welcome and encouraged throughout LSST. The modalities of feedback may evolve in Operations, but communication with the SCOC via the channels mentioned above is planned to continue. 

Rubin Observatory and the SCOC are infinitely grateful for the continuing contributions of the community to the design of the LSST Survey Strategy. The progress made in the past 10 years has led to important expected science gains across all science areas, as demonstrated by the significant improvements in science metrics built by the community. The unprecedented involvement of the scientific community at large in the refinement of the LSST survey strategy has been and continues to be a formidable success and a shining example of constructive collaborative practices in the scientific community. 

Rubin Observatory is grateful for the work of the SCOC members who, over the last four years volunteered their service to the Observatory and to the scientific community at large.
 